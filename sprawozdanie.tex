\documentclass{article}

\usepackage[OT4]{polski}
\usepackage[utf8]{inputenc}
\usepackage[left=2cm,right=2cm,top=1.5cm,bottom=2cm,includeheadfoot,a4paper]{geometry}
\usepackage{listings}
\usepackage{hyperref}

\geometry{a4paper}
\linespread{1.5}
\frenchspacing
\renewcommand{\arraystretch}{1.4}

\title{
	Projektowanie i programowanie systemów internetowych II\\
	\Huge{Ice Cream Places}
}

\date{}

\begin{document}
	\maketitle

	\section {Opis funkcjonalny systemu}
    Free Parking to system mający na celu ułatwienie użytkownikom znalezienia darmowych miejsc parkingowych w Polsce. Aplikacja zapewnia różne funkcje, które pomagają w tym celu. Dzięki temu użytkownicy mogą zaoszczędzić czas i pieniądze, a także uniknąć stresu związanego z szukaniem parkingu.
    
    \subsection{Aplikacja oferuje szereg funkcjonalności, takich jak:}
    \begin{itemize}
        \item \textbf{Autentykacja za pomocą e-maila i hasła} Użytkownicy mogą zarejestrować się w systemie, podając swój adres e-mail i hasło. Po rejestracji, mogą się logować do systemu za pomocą tych danych.
        \item \textbf{Autentykacja Google} Użytkownicy mają również możliwość rejestracji i logowania za pomocą swojego konta Google.
        \item \textbf{Mapa z lokalizacjami} System wyświetla mapę Polski, na której zaznaczone są lokalizacje darmowych parkingów. Użytkownicy mogą przeglądać mapę, aby znaleźć najbliższy dostępny parking.
        \item \textbf{Ulubione parkingi} Użytkownicy mogą dodawać parkingi do listy ulubionych, co ułatwia szybki dostęp do informacji o tych parkingach. Lista ulubionych parkingów jest dostępna w panelu bocznym.
        \item \textbf{Wyszukiwanie parkingów} Użytkownicy mogą wyszukiwać parkingi według miast.
        \item \textbf{Edycja profilu} Użytkownicy mogą zmieniać swoją nazwę użytkownika oraz zmieniać hasło.
        \item \textbf{Admin panel} Administrator może edytować parkingi i użytkowników.
    \end{itemize}

    \section{Streszczenie opisu technologicznego}
    Lista najważniejszych technologii i bibliotek, na bazie których powstała aplikacja:
    \begin{itemize}
        \item \textbf{Next.js} - framework dla Reacta, który umożliwia tworzenie aplikacji full-stack, rozszerzając najnowsze funkcje Reacta i integrując potężne narzędzia oparte na Rust do najszybszych kompilacji.
        \item \textbf{TypeScript} - silnie typowany język programowania, który buduje na JavaScript, dając lepsze narzędzia na dowolną skalę.
        \item \textbf{Prisma} - nowoczesny ORM dla Node.js i TypeScript, obsługujący PostgreSQL, MySQL itp. Zapewnia bezpieczeństwo typów, automatyczne migracje i intuicyjny model danych.
        \item \textbf{React} - biblioteka do tworzenia interfejsów użytkownika z indywidualnych elementów zwanych komponentami. 
        \item \textbf{React Hook Form} - prosta i łatwa biblioteka do walidacji formularzy, która redukuje ilość kodu, której potrzebujesz do napisania, jednocześnie eliminując niepotrzebne ponowne renderowania.
        \item \textbf{TanStack Query} - jest często opisywany jako brakująca biblioteka do pobierania danych dla aplikacji internetowych, ale z bardziej technicznego punktu widzenia sprawia, że pobieranie, buforowanie, synchronizowanie i aktualizowanie stanu serwera w aplikacjach internetowych jest dziecinnie proste.
        \item \textbf{React Icons} - pakiet, który pozwala na importowanie i używanie popularnych ikon z różnych bibliotek ikon w projektach React.
        \item \textbf{Zod} - biblioteka do deklarowania i walidacji schematów w TypeScript, z inferencją statycznego typu. 
        \item \textbf{ChakraUI} - biblioteka komponentów interfejsu użytkownika dla Reacta, umożliwiająca szybkie tworzenie atrakcyjnych i spójnych interfejsów.
        \item \textbf{GitHub} -  miejsce do kontroli wersji kodu. Pozwala na łatwą synchronizację między członkami grupy oraz na prostą kontrolę zmian w kodzie.
        \item \textbf{ESLint} -  narzędzie do analizy statycznej kodu JavaScript w celu wykrycia problemów i niezgodności ze standardami kodowania.
        \item \textbf{Prettier} -  narzędzie do formatowania kodu, zapewniające spójny styl kodowania w całym projekcie.
    \end{itemize}

    \section{Instrukcja lokalnego i zdalnego uruchomienia}
       \subsection{Lokalnie}
            \begin{itemize}
                \item Najpierw należy pobrać i zainstalować Node.js ze strony                 
                    \begin{lstlisting}
    https://nodejs.org/en/
                    \end{lstlisting}
                \item Następnie należy sklonować repozytorium za pomocą 
                    \begin{lstlisting}
    git clone https://github.com/parkingbezplatny/app.git
                    \end{lstlisting}
                \item Kolejno otworzyć terminal oraz przejść do folderu aplikacji oraz wpisać 
                    \begin{lstlisting}
    npm install              
                    \end{lstlisting}
                \item Po instalacji wystarczy wpisać w terminalu
                    \begin{lstlisting}
    npm run dev
                    \end{lstlisting} 
                \item Po chwili kompilacji powinna otworzyć się nasza aplikacja pod adresem 
                    \begin{lstlisting}
    http://localhost:3000/                        
                    \end{lstlisting}
            \item Aplikacja potrzebuje do działania pliku \textbf{.env}, w którym będą następujące informacje:
            \begin{itemize}
                \item \textbf{DATABASE\_URL} - URL do bazy danych
                \item \textbf{DIRECT\_URL} - direct URL do bazy danych
                \item \textbf{GOOGLE\_CLIENT\_ID} - clientId otrzymane od Google
                \item \textbf{GOOGLE\_CLIENT\_SECRET} - clientSecret otrzymany od Google
                \item \textbf{NEXTAUTH\_URL} - URL strony dla biblioteki NextAuth (http://localhost:3000)
                \item \textbf{NEXTAUTH\_SECRET} - klucz secret dla biblioteki NextAuth
                \item \textbf{NEXT\_PUBLIC\_MAP\_API\_KEY} - klucz do API dostawcy mapy
                \item \textbf{NEXT\_PUBLIC\_HERE\_API\_KEY} - klucz do API dostawcy geocodingu
            \end{itemize}
        \end{itemize}
            

            \subsection{Zdalnie}
            \begin{itemize}
                \item Strona internetowa aplikacji jest dostępna pod adresem 
                    \begin{lstlisting}
    https://freeparkingapp.vercel.app/
                    \end{lstlisting}
            \end{itemize}

            \subsection{Testy}
                \begin{itemize}
                    \item Po uruchomieniu lokalnie uruchamiając terminal w folderze z projektem wpisać
                        \begin{lstlisting}
    npm run test
                        \end{lstlisting}
                    \item Spis scenariuszy testowych wygenerowany po uruchomieniu testów:
                                    \begin{lstlisting}
PASS  __tests__/user/ChangePasswordModal.test.tsx
  ChangePasswordModal
    renders ChangePasswordModal correctly (762 ms)

PASS  __tests__/user/ChangeUsernameModal.test.tsx
  ChangeUsernameModal
    renders ChangeUsernameModal correctly (494 ms)

 PASS  __tests__/Sidepanel.test.tsx
  SidePanel
    renders SidePanel correctly (109 ms)
    handles click event correctly (34 ms)

 PASS  __tests__/AdminSidepanel.test.tsx
  AdminSidePanel
    renders AdminSidePanel correctly (104 ms)

 PASS  __tests__/WelcomePage.test.tsx
  WelcomePage
    redirects to dashboard if user is authenticated (104 ms)
    renders the WelcomePage component correctly (45 ms)

 PASS  __tests__/Navbar.test.tsx
  Navbar
    renders Navbar correctly (157 ms)
    handles click event correctly (104 ms)

 PASS  __tests__/MapTooltip.test.tsx
  MapTooltip
    renders MapTooltip correctly (71 ms)

 PASS  __tests__/Search.test.tsx
  Search
    renders Search correctly (122 ms)
    handles error correctly (55 ms)

 PASS  __tests__/AdminCard.test.tsx
  AdminCard
    renders AdminCard component correctly (59 ms)
    renders AdminCard component without footer (14 ms)

Test Suites: 9 passed, 9 total
Tests:       14 passed, 14 total
Snapshots:   0 total
Time:        17.822 s, estimated 18 s

                \end{lstlisting}
            \end{itemize}

    

    \section{Linki do dokumentacji projektu}
    \begin{itemize}
        \item \textbf{GitHub} - \href{https://github.com/parkingbezplatny/app}{https://github.com/parkingbezplatny/app}
    \end{itemize}
    \section{Wnioski projektowe}
    Podczas realizacji projektu systemu Free Parking, napotkaliśmy na kilka wyzwań. Najbardziej wymagającym elementem było zaimplementowanie mapy oraz zarządzanie dużą ilością danych o parkingach. Te zadania wymagały skomplikowanych rozwiązań technicznych i dużej ilości czasu na analizę i optymalizację.
Mimo tych wyzwań, naszym głównym celem było stworzenie intuicyjnego i prostego interfejsu dla użytkowników. Dzięki ciężkiej pracy i zaangażowaniu zespołu, udało nam się osiągnąć ten cel. Interfejs jest łatwy w obsłudze, a funkcje, takie jak wyszukiwanie parkingów czy dodawanie do ulubionych, są proste i intuicyjne.
Podsumowując, projekt zakończył się sukcesem. Współpraca w zespole była owocna, a wynikiem naszej pracy jest funkcjonalny i przyjazny dla użytkownika system.


\end{document}